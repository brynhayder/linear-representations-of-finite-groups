\documentclass[a4paper, oneside, 10pt]{article}
\usepackage[margin=2.5cm]{geometry}
\usepackage{amsmath}
\usepackage{exercise}


\usepackage{amssymb}
\usepackage{amsthm}
\usepackage{bm}
\usepackage{caption}
\usepackage[inline, shortlabels]{enumitem}
\usepackage{graphicx}
\usepackage{mathtools}
\usepackage{setspace}
\usepackage{dsfont}
\usepackage{xcolor}
\usepackage{upquote}
\usepackage{booktabs}
\usepackage{multirow}
\usepackage[section]{placeins}
\usepackage[utf8]{inputenc}
\usepackage[colorlinks=true, linkcolor=black, urlcolor=blue, citecolor=green, linktoc=all]{hyperref}
\usepackage{parskip}
\usepackage{verbatim}
\usepackage{breakcites}
\usepackage{centernot}
\usepackage[color=white]{todonotes}

\numberwithin{Answer}{section}
\numberwithin{Exercise}{section}


/Users/bryn/Programming/repos/dotfiles/texfiles/declarations.tex

\title{Solutions to Linear Representations of Finite Groups}
\author{% apparently can use authblk package to fix this
    Bryn Elesedy\thanks{\href{mailto:bryn@robots.ox.ac.uk}{\texttt{bryn@robots.ox.ac.uk}}}\\
Department of Computer Science\\
University of Oxford}
\date{\today}

\begin{document}
\maketitle
\tableofcontents
\clearpage
\setcounter{section}{1}

% should I put the questions in too???
\section{Character Theory}
\subsection{}
Let $V$ and $V'$ be the corresponding representations.
Then $\chi + \chi'$ is the character of the direct sum $V \oplus V'$.
The character of the alternating square is then, for any $s$ in the group,
\begin{align*}
    (\chi + \chi')_{\sigma}^2
    &= \frac12 ((\chi(s) + \chi'(s))^2 + \chi(s^2) +\chi'(s^2)) \\
    &= \frac12(\chi(s)^2 + \chi(s^2)) + \frac12(\chi'(s)^2 + \chi'(s^2)) + \chi(s) \chi'(s)\\
    &= \chi_{\sigma}^2(s) + {\chi_{\sigma}'}^2(s) + \chi(s)\chi'(s).
\end{align*}
We can do the same thing for the symmetric square
\begin{align*}
    (\chi + \chi')_{\alpha}^2
    &= \frac12 ((\chi(s) + \chi'(s))^2 - \chi(s^2) +\chi'(s^2)) \\
    &= \frac12(\chi(s)^2 - \chi(s^2)) + \frac12(\chi'(s)^2 - \chi'(s^2)) + \chi(s) \chi'(s)\\
    &= \chi_{\alpha}^2(s) + {\chi_{\alpha}'}^2(s) + \chi(s)\chi'(s).
\end{align*}

\subsection{}
In the permutation representation, the matrix $\rho_s$ will be a permutation matrix, 
with 1 at position $(i, i)$ if $i$ is fixed by $s$ and 0 on the diagonal otherwise.
The character $\chi(s)$ is the trace of this matrix, so is the number of elements fixed by $s$.

\subsection{}
We may choose bases $\{\vec{e}_i\}$ for $V$ and $\{\vec{e}_j'\}$ $V'$ such that 
$\langle\vec{e}_i, \vec{e}_j'\rangle = \delta_{ij}$.
Then for any matrix $A$ we have, in the given bases,
\[
    \langle\rho_s x, Ax'\rangle = \sum_{ij}x_i x_j' (\rho_s)_{ki}A_{kj}.
\]
So the map is invariant if and only if $A = (\rho_s^{-1})^\top$. 
We can then define a representation by $\rho_s' = (\rho_s^{-1})^\top$, 
which is clearly a representation.
Existence and uniqueness have therefore been established.

\subsection{}
The map $\rho_s$ is clearly linear and invertible, since the $\rho_{i, s}$ are linear and invertible.
Further, for any $f$ in $W$
\[
    \rho_{st}f = \rho_{2, st} f \rho_{1, st}^{-1} = \rho_{2, s} \rho_{2, t} f \rho_{1, t}^{-1} \rho_{1, s}^{-1} = \rho_{s} \rho_t f. 
\]
We can then calculate the character by expanding in a basis.
Let $f$ have components $f_{ab}$ in a basis for $W$.
Then (using Einstein notation)
\[
    (\rho_s f)_{ij} = (\rho_{2, s})_{i\alpha}(\rho_{1s}^{-1})_{\beta j} f_{\alpha \beta}.
\]
We calculate the trace of $\rho_s$ by contraction of $\rho_s \vec{e}_{ij}$ with elements
$\vec{e}_{kl}'$ of the dual basis, which are such that $\langle \vec{e}_{ij}, \vec{e}_{kl}'\rangle = \delta_{ij}\delta_{kl}$.
The conclusion then follows from $\chi(s^{-1}) = \chi(s)^*$ and that $\rho_{1, s}$ is a homomorphism.

\subsection{}
The character of the unit representation is $1$, so the number of times this representation
occurs in $\rho$ is therefore $(\chi | 1)= \frac{1}{\abs{G}} \sum_{g\in G} \chi(g)$.

\subsection{}
\begin{enumerate}[a)]
    \item Let $\vec{e}_x$ be the basis for the permutation representation $V$. 
        From each of the $c$ orbits, take a representative arbitrarily and denote them
        $y_i$ $i=1, \dots, c$. For each orbit, we can construct an invariant in $V$
        \[
            \nu_i = \sum_{g\in G} \rho_g \vec{e}_{y_i}.
        \]
        Denote by $W_i$ the space spanned by $\nu_i$, which is clearly invariant under G and irreducible,
        $W_i$ is fixed by $G$ and is hence the unit representation.
        We can therefore decompose $V$ into the direct sum
        \[
            U \oplus W_1 \oplus \dots \oplus W_c.
        \]
        We can notice that $U$ cannot contain the unit representation. If it did, then
        there would have to be an element of $U$ that is fixed by $G$, but this element would 
        form n orbit, which contradicts the hypothesis.
    \item In the permutation representation $\rho_s$ is a permutation matrix, so the character of
        the representation $\chi(s)$ is the number of elements of $X$ that are fixed by $s$. 
        Clearly, there must be $\chi(s)^2$ elements fixed by $\rho_s$ in $X\times X$.
    \item
    \begin{itemize}
        \item i) $\iff$ ii) $G$ is doubly transitive on $X\times X$, so is transitive on $X$, so the diagonal $\Delta$ is an orbit.
            Since $G$ is doubly transitive, $\Delta^\text{c}$ is also an orbit. So these are the only two orbits. 
            These statements also work the other way around. 
        \item ii) $\iff$ iii) is just part a)
        \item iii) $\iff$ iv) is given in the hint. $\chi = 1 + \psi$ where $\psi$ is the character of $\theta$.
            Then since $(\chi^2 | 1) = 2$ (two orbits) expanding the sqare tells us that $(\psi^2|1) = 1$. 
            However, because $\psi$ is real valued (all of the matrices will be real in the permutation representation)
            we see that $(\psi^2|1) = \frac1{\abs{G}} \sum_{g\in G} \psi(s) \psi(s)^* = (\psi|\psi)$. 
            We know that if a character $\eta$ has $(\eta | \eta) = 1$ then it is the character of an irreducible representation,
            so $\theta$ is irreducible.
    \end{itemize}
\end{enumerate}

\subsection{}
Let $\chi$ be a character of $G$. 
We know that the regular representation has character $r_G(s) = \abs{G}\1{s=e}$.
If $\chi$ vanishes on all elements apart from the identity then we have that $\chi(s) = \chi(e) \1{s=e}$.
The number of times that $\chi$ contains the unit representation is the number of distinct orbits in $G$,
which is clearly an integer. Thus, with 1 denoting the character of the unit representation,
\[
    (\chi|1) = \frac1{\abs{G}} \sum_{g \in G} \chi(g) = \frac{\chi(e)}{\abs{G}} \in \Z
\]
so $\chi(e)$ is an integral multiple of $\abs{G}$.

\subsection{}\label{ex:2.8}
\begin{enumerate}[a)]
    \item We can write $V_i = \bigoplus_{j=1}^{m_i} W_i$ where the $W_i$ are irreducible representations.
        The map $h_i : W_i \to V_i$ can be used to define a linear map $h_i^j: W_i \to W_i$ with image the 
        $j^{\text{th}}$ copy of $W_i$ by composing with a linear projection. This will of course still be linear
        and still commute with $\rho_s$ because of the direct sum. By Schur's lemma we get that each $h_j$ is a scalar
        multiple of the identity. For any $h_i$ we therefore have $\dim V / \dim W_i$ choices of these scalars,
        so this is the dimension of $H$.
    \item $a(h \cdot w) = (ah)\cdot w = h \cdot (aw)$ by the definition of
        tensor product multiplication. Hence $F$ is linear by linearity of $a$ and the sum
        given.
        Recall that $\rho$ acts trivially on $h$ so we have 
        \[
            F( \rho_s (h \cdot w)) = (\rho_s h) (\rho_s w) = \rho_s h (w) = \rho_s F (h \cdot w)
        \]
        because $h$ commutes with $\rho_s$.
        Clearly $F$ is non-zero, otherwise $H$ would be trivial. By Schur's lemma we therefore know that
        $H\otimes W_i \cong V_i$ and that $F$ is a scalar multiple of the identity, hence provides us with
        such an isomorphism.
    \item \emph{I'm not totally clear on what this question means.} I think `in the obvious way' means
        $h(\vec{w}) = \sum_\alpha h_\alpha w_\alpha$, in which case we can just apply part b) and we are done.
        Another possible interpretation is that $h(\vec{w}) = \sum_{\alpha} a_\alpha h_\alpha (w_\alpha)$ for some
        scalars $a_\alpha$, but in this case some of the scalars could be $0$ and I think that this would
        mean that $h$ is not invertible and hence couldn't be an isomorphism.
\end{enumerate}

\subsection{}
$V_i = \bigoplus_{j=1}^m W_i$ and $V_{i, \alpha}$ is the image projection of the projection $p_{\alpha \alpha}$.
We know from the proof of proposition 8 that the action of $p_{\alpha \alpha}$ on the basis of $W_k$ $\{e^k_\alpha\}$
is $p_{\alpha \alpha}e^k_\beta = \1{\beta = \alpha} e^k_\alpha$, so $V_{i, \alpha}$ is the direct sum
\[
    V_{i, \alpha} = \bigoplus_{j=1}^m \vecspan \,\{e^j_\alpha\}.
\]
We also know from~\ref{ex:2.8} that $h$ can be written in terms of a basis $h_i$ where
each of the $h_i = \lambda_i I$ is a scalar multiple of the identity.
This means that $h$ is surjective, since each summand of the direct sum is spanned by one element.
It also means that $h$ is injective because the sum is direct, so the elements of 
each of the summand are linearly independent.
By linearity each $h$ is clearly a homomorphism.
The proof is complete.

\subsection{}
\emph{Don't understand this exercise and can't finish it\dots}
\begin{itemize}
    \item $V(x)$ is the smallest sub rep of $V$ that contains $x$. This means that $V(x)$ is the subspace
        generated by the orbit of $x$ under $G$
        \[
            V(x) = \vecspan{} \{\rho_s x: s \in G\}.
        \]
    \item We can write $V_i = \bigoplus_{j=1}^m W_i$ where the $j^{\text{th}}$ $W_i$ has basis $\{e^j_\alpha\}$.
        So $x^1_\alpha = \sum_{j=1}^m x_{j, \alpha} e^j_1$ where $x_{j, \alpha}$ are the components of 
        the $j^{\text{th}}$ term in the direct summand that forms $x \in V$ in the appropriate basis.

\end{itemize}


\section{Subgroups, Products and Induced Representations}
\subsection{}
Let $\rho$ be a representation of $G$ corresponding to the irreducible representation space $V$.
$G$ is Abelian, therefore for any $s, t \in G$
\[
    \rho_t \rho_s = \rho_{ts} = \rho_{st} = \rho_{s} \rho_{t}.
\]
An application of Schur's lemma then tells us that $\rho_s = \lambda_s I$ is a homothety for 
any element $s$ of $G$. 
$V$ therefore must have dimension 1, since if it has large dimension then choosing an orthogonal basis
for this space exhibits a direct sum decomposition of $V$ into subspaces stable under this representation of $G$. 

\subsection{}
\begin{enumerate}[a)]
    \item The center $C$ of $G$ is an Abelian subgroup, so we know from the solution to the previous exercise
        that $\rho$ is a homothety for any $s\in C$. Write $\rho_s = \lambda_s I$, then $\chi(s) = n \lambda_s$
        and 
        \[
            \abs{\chi(s)}^2 = \chi(s) \chi(s)^* = \chi(s) \chi(s^{-1}) = n^2 \lambda_s \lambda_s^{-1} = n^2.
        \]
        Note that $\lambda_s$ cannot be zero because $\rho$ must be invertible. 
    \item 
        \[
            \abs{G} = \sum_{s \in G} \abs{\chi(s)}^2 \ge \sum_{s \in C} \abs{\chi(s)}^2 = n^2c
        \]
        using part a).
    \item \mbox{}
        \begin{claim}
            $\rho$ faithful $\implies$ $\rho$ an injection.
        \end{claim}
        \begin{proof}
            If $\rho_s = \rho_g$ then $\rho_s^{-1} \rho_g = \rho_{s^{-1}} \rho_g = \rho_{s^{-1}g} = I$,
            which by the hypothesis means that $s^{-1}g = e$.
        \end{proof}
        Consider now the set of unit modulus complex scalars $\omega_s = \frac1n \chi(s)$.
        There are exactly $\abs{C}$ of these because $\rho$ is injective. 
        All of these satisfy $\omega_s^{\abs{C}} = 1$ because $C$ is a group.
        These are therefore the roots of unity, which form a cyclic group under multiplication.
\end{enumerate}

\subsection{}
Let's take the multiplication to mean that $\chi_1 \chi_2$ is the function on $G$
with values $\chi_1(s) \chi_2(s)$.
This operation is clearly commutative since it returns elements of the underlying
field of the representation space.
We are given that $\hat{G}$ is closed under multiplication.
From the previous exercises we know that the irreducible characters of $G$
are of the form $\chi(s) = n \lambda_s$ where $\lambda_s \ne 0$, so these are invertible.
This shows that $\hat{G}$ is an Abelian group.
Now, since $G$ is Abelian we know it has exactly $\abs{G}$ conjugacy classes and
therefore $\abs{G}$ irreducible characters.
Since $\hat{G}$ is closed, this means that $|\hat{G}| =\abs{G}$.


Write $\psi_x: \chi \mapsto \chi(x)$ for $x\in  G$. 
We are told that this is an irreducible character of $\hat{G}$.
Define multiplication on $\hat{\hat{G}}$ by $\psi_x \psi_y \chi = \chi(x) \chi(y)$.
We can verify that $\psi_x \psi_y \chi = \chi(x) \chi(y) = \lambda_x \lambda_y = \lambda_{xy} = \chi(xy)$,
since the irreducible representations have degree 1 because $\hat{G}$ is Abelian.
This means that $\hat{\hat{G}}$ is an Abelian group. 

Since $\hat{G}$ is an Abelian group of order $\abs{G}$, we know that it
has $\abs{G}$ irreducible characters (with equality up to isomorphism).
So the set $\hat{\hat{G}}$ has order at most $\abs{G}$.
If the map $s \mapsto \psi_s$ is injective then the set $\hat{\hat{G}}$
has size $\abs{G}$, and above we checked that this is a homomorphism.
An injective map between two finite sets of the same size is a bijection.
Hence, checking injectivity is enough to prove isomorphism.
\todo[inline]{check this and prove injectivity!}



\nocite{*}
\bibliographystyle{apalike}
\bibliography{references}
\end{document}
