\documentclass[a4paper, oneside, 10pt]{article}
\usepackage[margin=2.5cm]{geometry}
\usepackage{amsmath}
\usepackage{exercise}


\usepackage{amssymb}
\usepackage{amsthm}
\usepackage{bm}
\usepackage{caption}
\usepackage[inline, shortlabels]{enumitem}
\usepackage{graphicx}
\usepackage{mathtools}
\usepackage{setspace}
\usepackage{dsfont}
\usepackage{xcolor}
\usepackage{upquote}
\usepackage{booktabs}
\usepackage{multirow}
\usepackage[section]{placeins}
\usepackage[utf8]{inputenc}
\usepackage[colorlinks=true, linkcolor=black, urlcolor=blue, citecolor=green, linktoc=all]{hyperref}
\usepackage{parskip}
\usepackage{verbatim}
\usepackage{breakcites}
\usepackage{centernot}
\usepackage[color=white]{todonotes}

\numberwithin{Answer}{section}
\numberwithin{Exercise}{section}


/Users/bryn/Programming/repos/dotfiles/texfiles/declarations.tex

\title{Solutions to Linear Representations of Finite Groups}
\author{% apparently can use authblk package to fix this
    Bryn Elesedy\thanks{\href{mailto:bryn@robots.ox.ac.uk}{\texttt{bryn@robots.ox.ac.uk}}}\\
Department of Computer Science\\
University of Oxford}
\date{\today}

\begin{document}
\maketitle
\tableofcontents
\clearpage
\setcounter{section}{1}

% should I put the questions in too???
\section{Character Theory}
\subsection{}
Let $V$ and $V'$ be the corresponding representations.
Then $\chi + \chi'$ is the character of the direct sum $V \oplus V'$.
The character of the alternating square is then, for any $s$ in the group,
\begin{align*}
    (\chi + \chi')_{\sigma}^2
    &= \frac12 ((\chi(s) + \chi'(s))^2 + \chi(s^2) +\chi'(s^2)) \\
    &= \frac12(\chi(s)^2 + \chi(s^2)) + \frac12(\chi'(s)^2 + \chi'(s^2)) + \chi(s) \chi'(s)\\
    &= \chi_{\sigma}^2(s) + {\chi_{\sigma}'}^2(s) + \chi(s)\chi'(s).
\end{align*}
We can do the same thing for the symmetric square
\begin{align*}
    (\chi + \chi')_{\alpha}^2
    &= \frac12 ((\chi(s) + \chi'(s))^2 - \chi(s^2) +\chi'(s^2)) \\
    &= \frac12(\chi(s)^2 - \chi(s^2)) + \frac12(\chi'(s)^2 - \chi'(s^2)) + \chi(s) \chi'(s)\\
    &= \chi_{\alpha}^2(s) + {\chi_{\alpha}'}^2(s) + \chi(s)\chi'(s).
\end{align*}

\subsection{}
In the permutation representation, the matrix $\rho_s$ will be a permutation matrix, 
with 1 at position $(i, i)$ if $i$ is fixed by $s$ and 0 on the diagonal otherwise.
The character $\chi(s)$ is the trace of this matrix, so is the number of elements fixed by $s$.

\subsection{}
We may choose bases $\{\vec{e}_i\}$ for $V$ and $\{\vec{e}_j'\}$ $V'$ such that 
$\langle\vec{e}_i, \vec{e}_j'\rangle = \delta_{ij}$.
Then for any matrix $A$ we have, in the given bases,
\[
    \langle\rho_s x, Ax'\rangle = \sum_{ij}x_i x_j' (\rho_s)_{ki}A_{kj}.
\]
So the map is invariant if and only if $A = (\rho_s^{-1})^\top$. 
We can then define a representation by $\rho_s' = (\rho_s^{-1})^\top$, 
which is clearly a representation.
Existence and uniqueness have therefore been established.

\subsection{}
The map $\rho_s$ is clearly linear and invertible, since the $\rho_{i, s}$ are linear and invertible.
Further, for any $f$ in $W$
\[
    \rho_{st}f = \rho_{2, st} f \rho_{1, st}^{-1} = \rho_{2, s} \rho_{2, t} f \rho_{1, t}^{-1} \rho_{1, s}^{-1} = \rho_{s} \rho_t f. 
\]
We can then calculate the character by expanding in a basis.
Let $f$ have components $f_{ab}$ in a basis for $W$.
Then (using Einstein notation)
\[
    (\rho_s f)_{ij} = (\rho_{2, s})_{i\alpha}(\rho_{1s}^{-1})_{\beta j} f_{\alpha \beta}.
\]
We calculate the trace of $\rho_s$ by contraction of $\rho_s \vec{e}_{ij}$ with elements
$\vec{e}_{kl}'$ of the dual basis, which are such that $\langle \vec{e}_{ij}, \vec{e}_{kl}'\rangle = \delta_{ij}\delta_{kl}$.
The conclusion then follows from $\chi(s^{-1}) = \chi(s)^*$ and that $\rho_{1, s}$ is a homomorphism.

\subsection{}
The character of the unit representation is $1$, so the number of times this representation
occurs in $\rho$ is therefore $(\chi | 1)= \frac{1}{\abs{G}} \sum_{g\in G} \chi(g)$.

\subsection{}
\begin{enumerate}[a)]
    \item Let $\vec{e}_x$ be the basis for the permutation representation $V$. 
        From each of the $c$ orbits, take a representative arbitrarily and denote them
        $y_i$ $i=1, \dots, c$. For each orbit, we can construct an invariant in $V$
        \[
            \nu_i = \sum_{g\in G} \rho_g \vec{e}_{y_i}.
        \]
        Denote by $W_i$ the space spanned by $\nu_i$, which is clearly invariant under G and irreducible,
        $W_i$ is fixed by $G$ and is hence the unit representation.
        We can therefore decompose $V$ into the direct sum
        \[
            U \oplus W_1 \oplus \dots \oplus W_c.
        \]
        We can notice that $U$ cannot contain the unit representation. If it did, then
        there would have to be an element of $U$ that is fixed by $G$, but this element would 
        form n orbit, which contradicts the hypothesis.
    \item In the permutation representation $\rho_s$ is a permutation matrix, so the character of
        the representation $\chi(s)$ is the number of elements of $X$ that are fixed by $s$. 
        Clearly, there must be $\chi(s)^2$ elements fixed by $\rho_s$ in $X\times X$.
    \item
    \begin{itemize}
        \item i) $\iff$ ii) $G$ is doubly transitive on $X\times X$, so is transitive on $X$, so the diagonal $\Delta$ is an orbit.
            Since $G$ is doubly transitive, $\Delta^\text{c}$ is also an orbit. So these are the only two orbits. 
            These statements also work the other way around. 
        \item ii) $\iff$ iii) is just part a)
        \item iii) $\iff$ iv) is given in the hint. $\chi = 1 + \psi$ where $\psi$ is the character of $\theta$.
            Then since $(\chi^2 | 1) = 2$ (two orbits) expanding the sqare tells us that $(\psi^2|1) = 1$. 
            However, because $\psi$ is real valued (all of the matrices will be real in the permutation representation)
            we see that $(\psi^2|1) = \frac1{\abs{G}} \sum_{g\in G} \psi(s) \psi(s)^* = (\psi|\psi)$. 
            We know that if a character $\eta$ has $(\eta | \eta) = 1$ then it is the character of an irreducible representation,
            so $\theta$ is irreducible.
    \end{itemize}
\end{enumerate}

\subsection{}
Let $\chi$ be a character of $G$. 
We know that the regular representation has character $r_G(s) = \abs{G}\1{s=e}$.
If $\chi$ vanishes on all elements apart from the identity then we have that $\chi(s) = \chi(e) \1{s=e}$.
The number of times that $\chi$ contains the unit representation is the number of distinct orbits in $G$,
which is clearly an integer. Thus, with 1 denoting the character of the unit representation,
\[
    (\chi|1) = \frac1{\abs{G}} \sum_{g \in G} \chi(g) = \frac{\chi(e)}{\abs{G}} \in \Z
\]
so $\chi(e)$ is an integral multiple of $\abs{G}$.

\subsection{}

\nocite{*}
\bibliographystyle{apalike}
\bibliography{references}
\end{document}
